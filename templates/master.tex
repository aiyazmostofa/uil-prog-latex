\documentclass[12pt]{article}
\usepackage[letterpaper, width=7in, height=9in]{geometry}
\setcounter{secnumdepth}{-\maxdimen}
\usepackage{listings}
\lstset{
  basicstyle=\ttfamily
}
\begin{document}
% Start title page
\vspace{\fill}
\begin{center}
	\begin{huge}
		\textbf{ {{ .Title }} }
	\end{huge}
	
	\vspace{\fill}
	{{ .Subtitle }}
\end{center}
\vspace{\fill}
\subsection{General Notes}
\begin{itemize}
	\item Do the problems in any order you like.
	      They do not have to be done in order from 1 to {{ len .Problems }}.
	\item All the problems have a value of 60 points.
	\item There is no extraneous input.
	      All input is exactly as specified in the problem.
	      Unless specified by the problem, integer inputs will not have leading zeros.
	      Unless otherwise specified, your program should read to the end of the file.
	\item Your program should not print extraneous output.
	      Follow the form exactly as given in the problem.
	\item A penalty of 5 points will be assessed each time that an incorrect solution is submitted.
	      This penalty will only be assessed if a solution is ultimately judged as correct.
\end{itemize}
\vspace{\fill}
\subsection{Names of Problems}
\begin{center}
	\begin{tabular}{ |c|c| }
		\hline
		Number               & Name         \\
		\hline
		{{ range $i, $v := .Problems }}
		Problem {{ .Index }} & {{ .Title }} \\
		\hline
		{{ end }}
	\end{tabular}
\end{center}
\vspace{\fill}
% End title page
{{ range $i, $v := .Problems }}
\newpage
{{ $v.Content }}
{{ end }}
\end{document}
